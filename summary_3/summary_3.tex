\documentclass[a4]{scrartcl}

% \usepackage[ngerman]{babel}
\usepackage[utf8]{inputenc}
\usepackage{mathtools}
\usepackage{amsmath}
\usepackage{amssymb}
\usepackage{geometry}
\usepackage{scrlayer-scrpage}
\usepackage{float}
\pagestyle{scrheadings}
\clearscrheadfoot

\usepackage[backend=biber, maxbibnames=99]{biblatex}
\addbibresource{references.bib}

\setlength{\parindent}{0cm}


\geometry{
  paper=a4paper, % Change to letterpaper for US letter
  top=2cm, % Top margin
  bottom=1.5cm, % Bottom margin
  left=2cm, % Left margin
  right=3cm, % Right margin
}

\ohead{\\
Pina Kolling\\
piko0011}

\begin{document}


%-------------------------------------------------------------------




\section*{Summary: Lecture 3}

Summary for the chapters \textit{X} and \textit{X}. \cite{book}




%-------------------------------------------------------------------

\ \\
\\
\textbf{Notes in the lecture:} \\

Basic relations between complexity classes \\

Deterministic space includes nondeterministic time: \\
NTIME$(f(n)) \subseteq$ SPACE$(f(n)^2)$ \\
$d$ choices in every step (in $TM$): $1,...,d$ \\
fill something with $1$ in first step \\
second step: simulate nondeterminstic $TM$ \\
pick something and simulate it? % he is drawing boxes on the board...
Until we get to $d$ because we increment by $1$ in each step. \\

The reachability method: \\
graphs/graph edges are constructed \\
$M$ empties the tape and puts all the heads to the start \\
there is only a single node that is accepting \\

Savitch's theorem: \\
complexit function is at least $\log n$ \\
we are doing an intuitive sketch now \\
this theorem grabs some internal node $k$, check recursively if there is a path from $1$ to $k$ and from $k$ to $n$ \\
test if path from $1$ to $k$ with picking a midpoint again... \\
we can have $\log n$ many segments to work on \\
PATH(startnode, endnode, pathlength) checks if there is a path from startnode to endnode with the length pathlength (?) \\

Analysis: \\
imagine the graph \\
graph can be too large to construct \\
why does the meaning of $n$ change?













%-------------------------------------------------------------------

\newpage

\printbibliography




\end{document}