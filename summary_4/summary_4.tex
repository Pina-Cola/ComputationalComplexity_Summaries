\documentclass[a4]{scrartcl}

% \usepackage[ngerman]{babel}
\usepackage[utf8]{inputenc}
\usepackage{mathtools}
\usepackage{amsmath}
\usepackage{amssymb}
\usepackage{geometry}
\usepackage{scrlayer-scrpage}
\usepackage{float}
\pagestyle{scrheadings}
\clearscrheadfoot

\usepackage[backend=biber, maxbibnames=99]{biblatex}
\addbibresource{references.bib}

\setlength{\parindent}{0cm}

\usepackage{xcolor}


\geometry{
  paper=a4paper, % Change to letterpaper for US letter
  top=2cm, % Top margin
  bottom=1.5cm, % Bottom margin
  left=2cm, % Left margin
  right=3cm, % Right margin
}

\ohead{\\
Pina Kolling\\
piko0011}



\usepackage[framemethod=TikZ]{mdframed}

% Style %
\mdfdefinestyle{enviStyle}{
   innertopmargin = 10pt,
  linewidth      = 1pt,
  frametitlerule = true,
  roundcorner    = 2pt%
}


\newenvironment{CountingDefinition}[2][]{%
   \ifstrempty{#1}%
   {\mdfsetup{%
      frametitle={{\strut ~}}}
   }%
   {\mdfsetup{%
      frametitle={{\strut ~#1}}}%
   }%
   \mdfsetup{
      nobreak                   = true,
     linecolor                 = gray,
    frametitlebackgroundcolor = gray!50,
    style                     = enviStyle
   }
   \begin{mdframed}[]\relax%
   \label{#2}}{\end{mdframed}}
   
   
   
   

\begin{document}

\section*{Summary: Lecture 4}

Summary for the chapter \textit{7.3} from page 150 on. \cite{book}

\begin{CountingDefinition}[Nondeterministic Turing Machine]{def:validLabelPlacement}
A nondeterministic Turing machine (\textit{NTM}) has states, which have more than one possible next state for an action. The states are not completely determined by its action and the current symbol it sees, (unlike a deterministic Turing Machine).

NTMs are for example used in thought experiments. One of the most important problems in is the P versus NP problem: How difficult it is to simulate nondeterministic computation with a deterministic computer? \cite{JE, book}
\end{CountingDefinition}


\subsection*{Asymmetry of non-deterministism}


\textbf{Asymmetry of nondeterministic acceptance:}
\begin{itemize}
\item Example: find out if a formula $\varphi$ is satisfiable ($\varphi \in SAT$): 
\begin{itemize}
\item choose truth values for the variables nondeterministically
\item check if they make $\varphi$ become true
\end{itemize}
\item this approach seems to be unpractical so check whether $\varphi$ is not satisfiable ($\varphi \in \overline{SAT}$)
\item Question whether NP $=$ coNP is a statement about \textit{all} options
\end{itemize}

\color{red} TODO with book \\
\color{black}

\textbf{Asymmetry of nondeterministic space:}
\begin{itemize}
\item Example: REACHABILITY $\in$ NL
\begin{itemize}
\item starting at start node 1
\item algorithm walks through nondeterminstically chosen edges $ \leq n$ times
\item only current position is remembered ($\log n$ space)
\item accepts if current node is node $n$
\end{itemize}
\item this approach seems to be unpractical so check if node $n$ is \textit{not} reachable from node 1
\end{itemize}

\color{red} TODO with book \\
\color{black}


\begin{CountingDefinition}[$\log n$ space]{def:validLabelPlacement}
A graph algorithm using $O(\log n)$ space stores a fixed number of pointers, independent of $n$, and manipulates them in some way. 
\cite{unknown}
\end{CountingDefinition}

%\begin{CountingDefinition}[$\log n$ time]{def:validLabelPlacement}
%$O(\log N)$: time grows linearly while $n$ grows exponentially. 
%\end{CountingDefinition}


\textbf{Nondeterministically computing functions}
\begin{itemize}
\item A nondeterminstic Turing Mashine $M$ computes a function $f$ if the the following hold for every input $x$:
\begin{itemize}
\item one of the computations of $M$ stops in the halting state $h$with the corret result $f(x)$ on the output tape
\item all computations that do not correctly output $f(x)$ stop instead in a \textit{no}-state (this path failed then)
\end{itemize}
\end{itemize}

\color{red} TODO with book \\
\color{black}



%to show that it is impossible to ... in SPACE was unsolved until the 80s \\
%$h$ for function (yes or no or $h$?) \\
%$h$ is yes \\
%everything that fails stops in stae with no \\


\color{violet} Questions: \\
Does this lead to the Haltingproblem? 
% Why is the current position remembering in $\log n$ space?
\color{black}


\begin{CountingDefinition}[Problem: GRAPH REACHABILITY]{def:validLabelPlacement}
Given a graph $G$ and two nodes $n_1, n_2 \in V$, is there  path from $n_1$ to $n_2$?

A graph $G=(V, E)$ is a finite set $V$ of nodes and a set $E$ of edges as node pairs.
\ \\

REACHIBILITY can be nondeterministically solved in space $\log n$.
\end{CountingDefinition}


\section*{Immerman-Szelepscènyi}

\textbf{Theorem (Counting problem):} \\
Given a graph $G$ and a start node $x$, the number of nodes that are reachable from $x$ in $G$ can nondeterministically be computed in space
$\log n$ (where $n$ is the number of nodes of $G$). \\



\begin{itemize}
\item can be non-deterministically solved
\item solving as an extension to REACHABILITY
\item counting of nodes that can \textit{not} be reached similar: substract result from $n$ \\
$\rightarrow$ counting problem and its complement are identical
\end{itemize}

\textbf{Algorithm:}
\begin{itemize}
\item nodes $1,...,n$ with start node $1$
\item $S(i)$ is the set of nodes which are reachable from the startnode with a pathlength of $i$
\begin{itemize}
\item $S(0)$ will contain node 1
\item $s(1)$ will contain all neighbours of 1
\end{itemize}

\item Algorithm consitsts out of 4 nested for-loops:

\begin{itemize}
\item outer for loop:
\begin{itemize}
\item computes number of nodes reachable from initial node (for loop with $k$ steps) as $|S(1)|, |S(2)|, ..., |S(n-1)|$
\item $|S(n-1)|$ is the desired answer ($n$ is the number of nodes)
\item $|S(0)| = 1$ (contains only start node)
\item $|S(k)|$ is computed after producting $|S(k-1)|$
\item in each step the previous set is overwritten with the next one because the space is limited
\end{itemize}

\item second for loop:
\begin{itemize}
\item it is computed how far the previous steps got and summed up how far it can get
\item a counter $l$ is initialized to 0
\item $l$ gets incremented for each node $u$ which is in $S(k)$ (in the end: $l = |S(k)|$)
\end{itemize}

\item third loop:
\begin{itemize}
\item deciding if node $u$ belongs to $S(k)$
\item iterating over all nodes $v \in V$ one by one to reuse space
\item if node $v$ is in $S(k-1)$, a counter $m$ is incremented \\
$m$ counts the members of $S(k-1)$ that were found so far
\item if $u = v$ or there is an edge from $u$ to $v$: $u \in S(k)$ \\
$\rightarrow$ variable \textit{reply} gets set to true \\
\item if end is reached:
\item $u \notin S(k)$ if end is reached and reply is false: \\
if $m  < |S(k-1)|$ not all members of $S(k-1)$ have been ecountered: return \textit{no}
\item else return \textit{reply} \\
\end{itemize}

\item fourth loop:
\begin{itemize}
\item checking whether $v \in S(k-1)$ with non-determinism (similar to REACHBILITY)
\end{itemize}



\end{itemize}

\item nodes can't be marked (this would use linear space) 
\item runs in space $\log n$ with a Truring Machine $M$
\item $M$ has seperate strings holding each of the variables: $k, |S(k-1)|, l, u, m, v, p, w_p, w_{p-1}, input, output$ \\
all of those need only to be compared to each other and incremented by 1 \\
all bounded by $n$ 


\end{itemize}


\color{violet} Questions:
\color{black}




\section*{$\overline{\text{REACHABILITY}} \in$ NL}

\begin{itemize}
\item NL $=$ nondeterministic logarithmic space
\item $\overline{\text{REACHABILITY}} \in$ NL
\end{itemize}


Modify the non-deterministic Turing Mashine from above so that it returns \textit{yes} if the innermost subroutine ever reaches the target node $n$, otherwise, return \textit{no}.

\begin{itemize}
\item run previous algorithm
\item if target node is found, yes is returned
\item else algorithm continues
\end{itemize}

\color{violet} Questions: \\
Did I understand this right?
\color{black}



\section*{NSPACE is closed under complement}

\begin{align*}
\text{NSPACE}(f(n)) = \text{coNSPACE}(f(n))
\end{align*}
for all proper complexity functions $f(n) \geq \log n$.


\color{red} TODO \\
\color{black}
\color{violet} Questions:
\color{black}




\newpage

\printbibliography




\end{document}