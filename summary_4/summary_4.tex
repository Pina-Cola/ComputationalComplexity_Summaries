\documentclass[a4]{scrartcl}

% \usepackage[ngerman]{babel}
\usepackage[utf8]{inputenc}
\usepackage{mathtools}
\usepackage{amsmath}
\usepackage{amssymb}
\usepackage{geometry}
\usepackage{scrlayer-scrpage}
\usepackage{float}
\pagestyle{scrheadings}
\clearscrheadfoot

\usepackage[backend=biber, maxbibnames=99]{biblatex}
\addbibresource{references.bib}

\setlength{\parindent}{0cm}

\usepackage{xcolor}


\geometry{
  paper=a4paper, % Change to letterpaper for US letter
  top=2cm, % Top margin
  bottom=1.5cm, % Bottom margin
  left=2cm, % Left margin
  right=3cm, % Right margin
}

\ohead{\\
Pina Kolling\\
piko0011}



\usepackage[framemethod=TikZ]{mdframed}

% Style %
\mdfdefinestyle{enviStyle}{
   innertopmargin = 10pt,
  linewidth      = 1pt,
  frametitlerule = true,
  roundcorner    = 2pt%
}


\newenvironment{CountingDefinition}[2][]{%
   \ifstrempty{#1}%
   {\mdfsetup{%
      frametitle={{\strut ~}}}
   }%
   {\mdfsetup{%
      frametitle={{\strut ~#1}}}%
   }%
   \mdfsetup{
      nobreak                   = true,
     linecolor                 = gray,
    frametitlebackgroundcolor = gray!50,
    style                     = enviStyle
   }
   \begin{mdframed}[]\relax%
   \label{#2}}{\end{mdframed}}
   
   
   
   

\begin{document}

\section*{Summary: Lecture 4}

Summary for the chapter \textit{7.3} from page 150 on. \cite{book}

\begin{CountingDefinition}[Nondeterministic Turing Machine]{def:validLabelPlacement}
A nondeterministic Turing machine (\textit{NTM}) has states, which have more than one possible next state for an action. The states are not completely determined by its action and the current symbol it sees, (unlike a deterministic Turing Machine).

NTMs are for example used in thought experiments. One of the most important problems in is the P versus NP problem: How difficult it is to simulate nondeterministic computation with a deterministic computer? \cite{JE, book}
\end{CountingDefinition}


\subsection*{Asymmetry of non-deterministism}


\textbf{Asymmetry of nondeterministic acceptance:}
\begin{itemize}
\item Example: find out if a formula $\varphi$ is satisfiable ($\varphi \in SAT$): 
\begin{itemize}
\item choose truth values for the variables nondeterministically
\item check if they make $\varphi$ become true
\end{itemize}
\item this approach seems to be unpractical so check whether $\varphi$ is not satisfiable ($\varphi \in \overline{SAT}$)
\item Question whether NP $=$ coNP is a statement about \textit{all} options
\end{itemize}

\color{red} TODO with book \\
\color{black}

\textbf{Asymmetry of nondeterministic space:}
\begin{itemize}
\item Example: REACHABILITY $\in$ NL
\begin{itemize}
\item starting at start node 1
\item algorithm walks through nondeterminstically chosen edges $ \leq n$ times
\item only current position is remembered ($\log n$ space)
\item accepts if current node is node $n$
\end{itemize}
\item this approach seems to be unpractical so check if node $n$ is \textit{not} reachable from node 1
\end{itemize}

\color{red} TODO with book \\
\color{black}


\begin{CountingDefinition}[$\log n$ space]{def:validLabelPlacement}
A graph algorithm using $O(\log n)$ space stores a fixed number of pointers, independent of $n$, and manipulates them in some way. 
\cite{unknown}
\end{CountingDefinition}

%\begin{CountingDefinition}[$\log n$ time]{def:validLabelPlacement}
%$O(\log N)$: time grows linearly while $n$ grows exponentially. 
%\end{CountingDefinition}


\textbf{Nondeterministically computing functions}
\begin{itemize}
\item A nondeterminstic Turing Mashine $M$ computes a function $f$ if the the following hold for every input $x$:
\begin{itemize}
\item one of the computations of $M$ stops in the halting state $h$with the corret result $f(x)$ on the output tape
\item all computations that do not correctly output $f(x)$ stop instead in a \textit{no}-state (this path failed then)
\end{itemize}
\end{itemize}

\color{red} TODO with book \\
\color{black}



%to show that it is impossible to ... in SPACE was unsolved until the 80s \\
%$h$ for function (yes or no or $h$?) \\
%$h$ is yes \\
%everything that fails stops in stae with no \\


\color{violet} Questions: \\
Does this lead to the Haltingproblem? 
% Why is the current position remembering in $\log n$ space?
\color{black}


\section*{Immerman-Szelepscènyi}

\textbf{Theorem:} \\
Given a graph $G$ and a start node $x$, the number of nodes that are reachable from $x$ in $G$ can nondeterministically be computed in space
$\log n$ (where $n$ is the number of nodes of $G$).
\ \\

\textbf{Proof concept:}
\begin{itemize}
\item nodes $1,...,n$ with start node $1$
\item $S(i)$ is the set of nodes which are reachable from the startnode with a pathlength of $i$
\begin{itemize}
\item $S(0)$ will contain node 1
\item $s(1)$ will contain all neighbours of 1
\end{itemize}

\item Algorithm consitsts out of 3 nested for-loops:

\begin{itemize}
\item outer for loop:
\begin{itemize}
\item computes number of nodes reachable from initial node (for loop with $k$ steps)
\item in each stepthe previous set is overwritten with the next one because the space is limited
\end{itemize}

\item second for loop:
\begin{itemize}
\item it is computed how far the previous steps got and summed up how far it can get
\end{itemize}

\item third loop:
\begin{itemize}
\item checking if node belongs to $S(i)$
\item return no when all guesses were correct? 
% we remember solution that we were supposed to reach beforehand
\end{itemize}
\end{itemize}

\item nodes cant be marked (this would use linear space) but with determinism it gets into $\log n$ 


\end{itemize}

Algorithm (2) slides seems to be important

\color{red} TODO and TODO with book \\
\color{black}
\color{violet} Questions:
\color{black}


\section*{REACHABILITY $\in$ NL}
NL $=$ nondeterministic logarithmic space \\
little bit of stuff between log n and constant but no interesting stuff \\

\color{red} TODO \\
\color{black}
\color{violet} Questions:
\color{black}



\section*{NSPACE is closed under complement}

\begin{align*}
\text{NSPACE}(f(n)) = \text{coNSPACE}(f(n))
\end{align*}
for all proper complexity functions $f(n) \geq \log n$.


\color{red} TODO \\
\color{black}
\color{violet} Questions:
\color{black}




\newpage

\printbibliography




\end{document}