\documentclass[a4]{scrartcl}

% \usepackage[ngerman]{babel}
\usepackage[utf8]{inputenc}
\usepackage{mathtools}
\usepackage{amsmath}
\usepackage{amssymb}
\usepackage{geometry}
\usepackage{scrlayer-scrpage}
\usepackage{float}
\pagestyle{scrheadings}
\clearscrheadfoot

\usepackage[backend=biber, maxbibnames=99]{biblatex}
\addbibresource{references.bib}

\setlength{\parindent}{0cm}


\geometry{
  paper=a4paper, % Change to letterpaper for US letter
  top=2cm, % Top margin
  bottom=1.5cm, % Bottom margin
  left=2cm, % Left margin
  right=3cm, % Right margin
}

\ohead{\\
Pina Kolling\\
piko0011}

\begin{document}

\section*{Summary: Lecture 4}

Summary for the chapters \textit{X} and \textit{X}. \cite{book}

\subsection*{Asymmetry of non-deterministic revisited}

to check if formular is non satisfiable \\
question whether NP $=$ coNP \\

similar isuues is non-deterministic space: \\
write down $n$ nodes on a ... non-det. will succeed if it is possible \\
to show that it is impossible to ... in SPACE was unsolved until the 80s \\
$h$ for function (yes or no or $h$?) \\
$h$ is yes \\
everything that fails stops in stae with no \\


\section*{Immerman-Szelepscènyi}

how many distinct nodes can be reached in a graph if you start from a graph $x$ \\
$s(0)$ will contain node 1 and $s(1)$ will contain all neighbours of 1 \\
we will have actual names \\
4 nested for loops and algorithm happens in the middle \\
outer for loop: \\
computes number of nodes reachable from initial node (for $k$ steps with $k$ as the interative thingy in the for loop) \\
in each step we override the previous set with the next one because we only have limited space \\
second loop: \\
we get how far we got in the previous steps and sum up how far we can get (because we can get previous set size?) \\
third loop: \\
the actual magic happens here: checking something \\
Aux sounds like a port for headphones \\
return no when all guesses were correct? we remember solution that we were supposed to reach beforehand \\







\newpage

\printbibliography




\end{document}