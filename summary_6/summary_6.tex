\documentclass[a4]{scrartcl}

% \usepackage[ngerman]{babel}
\usepackage[utf8]{inputenc}
\usepackage{mathtools}
\usepackage{amsmath}
\usepackage{amssymb}
\usepackage{geometry}
\usepackage{scrlayer-scrpage}
\usepackage{float}
\usepackage{xcolor}
\pagestyle{scrheadings}
\clearscrheadfoot

\usepackage[backend=biber, maxbibnames=99]{biblatex}
\addbibresource{references.bib}

\setlength{\parindent}{0cm}


\geometry{
  paper=a4paper, % Change to letterpaper for US letter
  top=2cm, % Top margin
  bottom=1.5cm, % Bottom margin
  left=2cm, % Left margin
  right=3cm, % Right margin
}

\ohead{\\
Pina Kolling\\
piko0011}

\usepackage[framemethod=TikZ]{mdframed}

% Style %
\mdfdefinestyle{enviStyle}{
   innertopmargin = 10pt,
  linewidth      = 1pt,
  frametitlerule = true,
  roundcorner    = 2pt%
}


\newenvironment{CountingDefinition}[2][]{%
   \ifstrempty{#1}%
   {\mdfsetup{%
      frametitle={{\strut ~}}}
   }%
   {\mdfsetup{%
      frametitle={{\strut ~#1}}}%
   }%
   \mdfsetup{
      nobreak                   = true,
     linecolor                 = gray,
    frametitlebackgroundcolor = gray!50,
    style                     = enviStyle
   }
   \begin{mdframed}[]\relax%
   \label{#2}}{\end{mdframed}}

\begin{document}

\section*{Summary: Lecture 6}

Summary for the chapter \textit{8.2}. \cite{CC, book}


\subsection*{Completeness}


\begin{CountingDefinition}[]{def:validLabelPlacement}
Let $C$ be a complexity class and let $L$ be a language in $C$. $L$ is called $C$\textit{-complete} if any language $L' \in C$ can be reduced to $L$.
\ \\

(Every language of a complexity class can be reduced to $L$.)
\end{CountingDefinition}



\begin{itemize}
\item reducitbility is transitive $\rightarrow$ problems are ordered by difficulty
\item complete problems can capture the difficulty of a class
\item problem is seen as cpmpletely understood if the problem is complete
\end{itemize}


\color{violet} \textbf{Question: }

Which problems can be reduced to a formal language? 
\begin{itemize}
\item[] \color{black}
\textsc{Sat} can be expressed as formal language. \cite{lang} \\
$\Rightarrow$ \textsc{Sat} can be reduced to a formal language. (?) \\
\textsc{Sat} is in NP. \cite{book}
\ \\

Because \textsc{Circuit Sat} can be reduced to \textsc{Sat}: \textsc{Circuit Sat} can be reduced to a formal language. (?)
\textsc{Circuit Sat} is NP-complete. \cite{wiki} \\
Any formal language $L \in$ NP can be reduced to \textsc{Circuit Sat}? OR the other way around?
\ \\

\begin{CountingDefinition}[Formal language]{def:validLabelPlacement}
Formal languages are abstract languages, which define the syntax of the words that get accepted by that language. It is a set of words that get accepted by the language and has a set of symbols that is called alphabet, which contains all the possible characters of the words. Those characters are called nonterminal symbols. \cite{CNF, language}
\ \\

\textbf{Kleene star} \\
The Kleene star $\Sigma^*$ of an alphabet $\Sigma$ is the set of all words that can be created through concatenation of the symbols of the alphabet $\Sigma$. The empty word $\epsilon$ is included. 
\ \\

\textbf{Formal language}\\
A formal language $L$ over an alphabet $\Sigma$ is a subset of the Kleene star of the alphabet: $L \subseteq \Sigma^*$
\end{CountingDefinition}

Where to set the line between lanuguage decisions and other problems? Can every problem be contrcuted as a formal language?
\ \\

Is everything that is reducable to \textsc{Sat} reducable to a formal language because of the transitivity?
\ \\

I assume it does not have an influence on the complexity of a problem if it can be expressed as a formal language? Are formal languages part of specific complexity classes?

\end{itemize}

\color{black}



\begin{CountingDefinition}[Closed under reduction]{def:validLabelPlacement}
The following complexity classes are all closed under reductions:
\begin{itemize}
\item[] \textsc{P} \ \ \  \textsc{NP} \ \ \ \textsc{coNP}
\ \ \ \textsc{L}
\ \ \  \textsc{NL}
\ \ \  \textsc{PSPACE}
\ \ \ \textsc{EXP}
\end{itemize}

A class $C$ is closed under reductions if whenever $L$ is reducible to $L'$ and $L' \in C'$, then $L \ in C'$.
\ \\

If a complete problem in $C$ belongs in a class $C' \subseteq C$, $C = C'$.

\end{CountingDefinition}


\begin{figure}[H]
\begin{center}
\includegraphics[scale=0.2]{PNP.jpg}
\hspace*{5em}
\includegraphics[scale=0.15]{classes.png}
\end{center}
\caption{P and NP sets \cite{PNPsets} and complexity classes \cite{classesPic}}
\end{figure}




\begin{itemize}

\item examples: 
\begin{itemize}
\item if an NP-complete language is in P, then NP = P 
\item if a P-complete language is in L, then P = L
\item if a P-complete language is in NL, then P = NL
\item no EXP-complete language can be in P
\end{itemize}

\end{itemize}

\subsection*{Table method -- time complexity}

\begin{itemize}
\item table method to understand time complexity
\item Turing Machine $M$ decides language $L$ on an input $x$
\item computation on input $x$ as $|x|^k \times |x|^k$ computation table
\item[$\rightarrow$] $|x|^k$ is the time bound
\item rows $i$ are the time steps (from $0$ to $|x|^k-1$)
\item columns $j$ are the string positions
\item $T_{ij}$ represents the content of position $j$ of the string of $M$ at time $i$ (after $i$ steps)
\end{itemize}

\newpage 
\textbf{Example:}

\begin{itemize}
\item Turing Machine $M$ deciding palindromes in time $n^2$
\end{itemize}

\begin{figure}[H]
\begin{center}
\includegraphics[scale=0.8]{palindrometm.png}
\end{center}
\caption{Binary palindrome Turing Machine $M$ contructed from the definition in example 2.3 in the book \cite{book}}
\end{figure}

\begin{itemize}
\item edges of the Turing machine: \\
(input symbol $\sigma$, symbol to replace input symbol on tape with, direction in which the head moves)
\end{itemize}

\begin{figure}[H]
\begin{center}
\includegraphics[scale=0.4]{palindrome.jpg}
\end{center}
\caption{Definition of binary palindrome Turing Machine $M$ in example 2.3 in the book \cite{book}}
\end{figure}

\newpage
On input $x=0110$ it results in the following computaion table:

\begin{figure}[H]
\begin{center}
\includegraphics[scale=0.5]{computation_table.jpg}
\end{center}
\caption{Computation table of binary palindrome Turing Machine $M$ on input $x=0110$ in the book \cite{book}}
\end{figure}

\begin{itemize}
\item $\sigma$ is the current input symbol
\item $q$ is the current state
\item $\sqcup$ is the clank symbol
\item $\triangleright$ is the first symbol
\end{itemize}

\color{violet} Questions: \\
Is the Turing Macheine constructed correctly? Does the \textit{no}-state need a double circle, because it halts or does only the accepting state get a double circle?
\color{black}




\subsection*{\textsc{P}-completeness of \textsc{CircuitValue}}

\begin{CountingDefinition}[Problem: \textsc{CircuitValue}]{def:validLabelPlacement}
The \textsc{CircuitValue} Problem is the problem of computing the output of a given Boolean circuit on a given input.
\ \\

In terms of time complexity, it can be solved in linear time (topological sort).
\end{CountingDefinition}

\begin{itemize}
\item \textsc{P}-complete
\end{itemize}

\textbf{Proof idea:}

\begin{itemize}
\item \textsc{CircuitValue} is in P (prerequisite for being P-complete)
\item show: any language $L \in $ P can be reduced to \textsc{CircuitValue}
\item $L$ is decided by Turing Machine $M$ in polynomial time $(n^k)$
\item show: there is a reduction $R$ and an input $x$ to $M$ \\
$R$ puts out a circuit $C$ without variable gates, whose value is true
only if $M$ accepts $x$
\item computation table $T$ of $M$
\item if $i=0$: value of $T_{i,j}$ is the $j$th symbol of $x$ or a $\sqcup$
\item if $j=0$: value of $T_{i,j}$ is a $\triangleright$
\item if $j=|x|^k-1$: value of $T_{i,j}$ is a $\sqcup$
\end{itemize}

\begin{figure}[H]
\begin{center}
\includegraphics[scale=0.3]{rowsmarked.png}
\end{center}
\caption{Initial rows marked on previous palindrome example table}
\end{figure}

\begin{itemize}
\item value of $T_{i,j}$ is the content of position $j$ of the string on time $i$
\item[$\rightarrow$] depends on the same position and neighbor position in the previous steo $i-1$: $T_{i-1,j-1}, T_{i-1,j}, T_{i-1,j+1}$
\item[] \begin{figure}[H]
\begin{center}
\includegraphics[scale=0.3]{depends.jpg}
\end{center}
\caption{Value of $T_{i,j}$ depends on values of $T_{i-1,j-1}, T_{i-1,j}, T_{i-1,j+1}$}
\end{figure}
\item $T_{i-1,j-1}, T_{i-1,j}, T_{i-1,j+1} \in \Sigma \Rightarrow$ cursor not at position $j-1, j, j+1$, this concludes to $T_{i,j}=T_{i-1,j}$ \\
\item set $\Gamma$ contains all symbols that can appear on the table (symbols from $\Sigma$ or symbol state combinations)
\item encode each symbol $\sigma \in \Sigma$ as a vector $s=(s_1,...,s_m)$ 
\begin{itemize}
\item the entries of the vector ($s_1,...,s_m$) are either 0 or 1
\item the vector hast $m=\log |\Gamma|$ entries
\end{itemize}
\item computation table can now be constructed as a table $S_{i,j,l}$ with binary entries (and $0 \leq i \leq n^k -1$ and $0 \leq j \leq n^k -1$ and $1 \leq l \leq m$)
\item each binary entry $S_{ijl}$ depends on previous entries $S_{i-1,j-1,l'}, S_{i-1,j,l'}, S_{i-1,j+1,l'}$, where $l'$ ranges over 1 to $m$
\item there are $m$ Boolean functions $F_1,...,F_m$ such that
\begin{align*}
S_{i,j,l}=F_l(S_{i-1,j-1,1},..., S_{i-1,j-1,m}, S_{i-1,j+1,1},...,S_{i-1,j+1,m})
\end{align*}
\item every boolean function can be renderes as a boolean circuit
\item[$\rightarrow$] there is a boolean circuit $C$ with $3m$ inputs and $m$ outputs that computes $T_{i,j}$ with given $T_{i-1,j-1}, T_{i-1,j}, T_{i-1,j+1}$ \\

\item reduction $R$ from $L$ to \textsc{CircuitValue}
\item for each input $x$, $R(x)$ consists of $(|x|^k-1) \cdot (|x|^k-2)$ copies of the circuit $C$ \\
$\rightarrow$ one for each entry in $T_{i,j}$
\item call $C_{i,j}$ the $(i,j)$th copy of $C$
\item $C_{i,j}$ depends on $C_{i-1,j-1}, C_{i-1,j}, C_{i-1,j+1}$ (if $i \geq 0$)
\item first row and first and last column known
\item other entries based on these entries
\item output circuit of $R(x)$ is $C_{|x|^k-1,1}$ \\
(final step and string position 1, assuming that string position 1 contains \textit{yes} or \textit{no}) \\

\item value of circuit $R(x)$ is only true if $x \in L$ because of equivalence of table structure
\item $R(x)$ is in $\log n$ space

\end{itemize}

\textsc{CircuitValue} without NOT remains P-complete:
\begin{itemize}
\item AND, OR, NOT gate in circuit
\item monotone circuit: does not have NOT gates
\item move NOT downwards with DeMorgan's law (put negation into bracktes and exchange operator) until inputs are changed
\end{itemize}

\textsc{MonotoneCircuitValue} is P-complete.







\subsection*{\textsc{CircuitSat} is \textsc{NP}-complete}

\begin{CountingDefinition}[Problem: \textsc{CircuitSat}]{def:validLabelPlacement}
The circuit satisfiability problem (\textsc{CircuitSat}) is the decision problem of determining whether a given Boolean circuit has an assignment of its inputs that makes the output true.
\ \\

Input: a Boolean circuit $C$ 
\ \\

Question: Is there a truth assignment which makes $C$ output the value true?
\end{CountingDefinition}

\begin{itemize}
\item cook's theorem: \textsc{Sat} is \textsc{NP}-complete
\end{itemize} 
\textbf{Proof idea:}
\begin{itemize}
\item \textsc{CircuitSat} reduces to \textsc{SAT}
\item Turing Machine decides circuit nondeterministically
\item describe a reduction $R$
\item a variable is added in the nondeterministic Turing Machine
\item check if one of the variables is tue: use this choice (?)
\item problem: can we set thiese variables such that the Turing Machine accepts?
\item answer corresponds direct to \textit{is there a choice of nd decisions such that the turing machine accepts?}
\item extremely direct reduction
\item \textsc{SAT} is \textsc{NP}-complete
\end{itemize}




\color{red} TODO \\
proof! \\
\color{violet} Questions:
\color{black}

\newpage

\begin{CountingDefinition}[Relation between complexity classes]{def:validLabelPlacement}
$N \subseteq NL \subseteq NC \subseteq P \subseteq NP \subseteq PSPACE$
\end{CountingDefinition}


\subsection*{NP-complete problems}

\begin{figure}[H]
\begin{center}
\includegraphics[scale=0.6]{NP_complete.png}
\end{center}
\caption{NP-complete problems in relation}
\end{figure}


\begin{itemize}
\item $k$-\textsc{SAT} for $k \geq 3$ is NP-complete
\end{itemize}

\begin{CountingDefinition}[\textsc{Circuit-SAT}]{def:validLabelPlacement}
The circuit satisfiability problem (\textsc{Circuit-SAT}) is the decision problem of determining whether a given Boolean circuit has an assignment of its inputs that makes the output true.
\end{CountingDefinition}


\begin{CountingDefinition}[\textsc{SAT}]{def:validLabelPlacement}
The \textsc{SAT} (satisfiability) problem is the problem of determining if there exists an interpretation that satisfies a given Boolean formula. \cite{GTI}
\end{CountingDefinition}


\begin{CountingDefinition}[\textsc{3-SAT}]{def:validLabelPlacement}
Like the \textsc{SAT} problem, \textsc{3-SAT} is determining the satisfiability of a formula in CNF where each clause is limited to at most three literals.
\end{CountingDefinition}


\begin{CountingDefinition}[\textsc{Clique}]{def:validLabelPlacement}
The \textsc{Clique} problem is the problem of finding cliques (subsets of vertices, all adjacent to each other, also called complete subgraphs) in a graph.
\end{CountingDefinition}


\begin{CountingDefinition}[\textsc{VertexCover}]{def:validLabelPlacement}
In graph theory, a \textsc{VertexCover} (sometimes \textsc{NodeCover}) of a graph is a set of vertices that includes at least one endpoint of every edge of the grap
\end{CountingDefinition}


\begin{CountingDefinition}[\textsc{HamiltonCycle}]{def:validLabelPlacement}
A \textsc{HamiltonCycle} is a graph cycle (i.e., closed loop) through a graph that visits each node exactly once.
\end{CountingDefinition}


\begin{CountingDefinition}[\textsc{TravellingSalesman}]{def:validLabelPlacement}
\textit{Given a list of cities and the distances between each pair of cities, what is the shortest possible route that visits each city exactly once and returns to the origin city?}
\end{CountingDefinition}


\begin{CountingDefinition}[\textsc{GraphColoring}]{def:validLabelPlacement}
In graph theory, graph coloring is a special case of graph labeling. It is an assignment of colors to elements of a graph subject to certain constraints.
\end{CountingDefinition}


\begin{CountingDefinition}[\textsc{ExactCover}]{def:validLabelPlacement}
Given a collection $S$ of subsets of set $X$, an exact cover is the subset $S^*$ of $S$ such that each element of $X$ is contained is exactly one subset of $S^*$.
\end{CountingDefinition}


\begin{CountingDefinition}[\textsc{Knapsack}]{def:validLabelPlacement}
\textit{Given a set of items, each with a weight and a value, determine the number of each item to include in a collection so that the total weight is less than or equal to a given limit and the total value is as large as possible.}
\end{CountingDefinition}


\begin{CountingDefinition}[\textsc{SubsetSum}]{def:validLabelPlacement}
The \textsc{SubsetSum} problem involves determining whether or not a subset from a list of integers can sum to a target value. For example, consider the list of $nums = [1, 2, 3, 4]$ . If the target is 7 , there are two subsets that achieve this sum: $\{ 3, 4 \}$ and $\{ 1, 2, 4\}$.
\end{CountingDefinition}





\subsection*{P-complete problems}

\begin{itemize}
\item \textsc{CircuitValue}
\item \textsc{LinearProgramming}
\item \textsc{HornSAT}
\end{itemize}

\begin{CountingDefinition}[\textsc{CircuitValue}]{def:validLabelPlacement}
The \textsc{CircuitValue} Problem is the problem of computing the output of a given Boolean circuit on a given input.

In terms of time complexity, it can be solved in linear time (topological sort).


The problem is closely related to the \textsc{Sat} (Boolean Satisfiability) problem which is complete for NP and its complement, which is complete for co-NP.
\end{CountingDefinition}


\begin{CountingDefinition}[\textsc{LinearProgramming}]{def:validLabelPlacement}
\textsc{LinearProgramming} is a method to achieve the best outcome (such as maximum profit or lowest cost) in a mathematical model whose requirements are represented by linear relationships. 
\end{CountingDefinition}


\begin{CountingDefinition}[\textsc{HornSAT}]{def:validLabelPlacement}
\textsc{HornSAT} is the problem of deciding whether a given set of propositional Horn clauses is satisfiable or not. 

A Horn clause is a clause (a disjunction of literals) with at most one positive literal.
\end{CountingDefinition}




\subsection*{NL problems}

\begin{itemize}
\item \textsc{2-Sat}
\item \textsc{Reachability}
\end{itemize}


\begin{CountingDefinition}[\textsc{2-Sat}]{def:validLabelPlacement}
Like the \textsc{SAT} and \textsc{3-SAT} problem, \textsc{2-SAT} is determining the satisfiability of a formula in CNF where each clause is limited to at most two literals.
\end{CountingDefinition}


\begin{CountingDefinition}[\textsc{Reachability}]{def:validLabelPlacement}
Given a graph $G$ and two nodes $n_1, n_2 \in V$, is there  path from $n_1$ to $n_2$?

A graph $G=(V, E)$ is a finite set $V$ of nodes and a set $E$ of edges as node pairs.
\ \\

\textsc{Reachability} can be nondeterministically solved in space $\log n$.
\end{CountingDefinition}




\subsection*{L problems}

\begin{itemize}
\item \textsc{1-Sat}
\end{itemize}


\begin{CountingDefinition}[\textsc{1-Sat}]{def:validLabelPlacement}
Like the \textsc{SAT} and \textsc{3-SAT} problem, \textsc{2-SAT} is determining the satisfiability of a formula in CNF where each clause is limited to at most one literal.
\end{CountingDefinition}





\newpage

\printbibliography




\end{document}