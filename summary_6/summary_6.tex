\documentclass[a4]{scrartcl}

% \usepackage[ngerman]{babel}
\usepackage[utf8]{inputenc}
\usepackage{mathtools}
\usepackage{amsmath}
\usepackage{amssymb}
\usepackage{geometry}
\usepackage{scrlayer-scrpage}
\usepackage{float}
\usepackage{xcolor}
\pagestyle{scrheadings}
\clearscrheadfoot

\usepackage[backend=biber, maxbibnames=99]{biblatex}
\addbibresource{references.bib}

\setlength{\parindent}{0cm}


\geometry{
  paper=a4paper, % Change to letterpaper for US letter
  top=2cm, % Top margin
  bottom=1.5cm, % Bottom margin
  left=2cm, % Left margin
  right=3cm, % Right margin
}

\ohead{\\
Pina Kolling\\
piko0011}

\usepackage[framemethod=TikZ]{mdframed}

% Style %
\mdfdefinestyle{enviStyle}{
   innertopmargin = 10pt,
  linewidth      = 1pt,
  frametitlerule = true,
  roundcorner    = 2pt%
}


\newenvironment{CountingDefinition}[2][]{%
   \ifstrempty{#1}%
   {\mdfsetup{%
      frametitle={{\strut ~}}}
   }%
   {\mdfsetup{%
      frametitle={{\strut ~#1}}}%
   }%
   \mdfsetup{
      nobreak                   = true,
     linecolor                 = gray,
    frametitlebackgroundcolor = gray!50,
    style                     = enviStyle
   }
   \begin{mdframed}[]\relax%
   \label{#2}}{\end{mdframed}}

\begin{document}

\section*{Summary: Lecture 6}

Summary for the chapter \textit{8.2}. \cite{book, CC}

\begin{CountingDefinition}[Title]{def:validLabelPlacement}
Content
\end{CountingDefinition}

\subsection*{Completeness}

\begin{itemize}
\item every language of a complexity class can be reduced to $L$ $\rightarrow$ you only need what $L$ describes
\end{itemize}

\subsection*{What does completeness do for us?}

\begin{itemize}
\item a reduction definition is usefull because the complexity classe are closed under reduction
\item examples look helpfull
\item $L$ and $R$ seem to be important:
\begin{align*}
L' \in P \ \ \ \ \ \ \ & A \\
L \rightarrow L' \ \ \ \ \ \ \  & R
\end{align*}

\item drawing set circle inclusion thing (P and NP)

\end{itemize}

\subsection*{P-completeness of CIRCUIT VALUE}

\begin{CountingDefinition}[Problem: CIRCUIT VALUE]{def:validLabelPlacement}
The CIRCUIT VALUE Problem is the problem of computing the output of a given Boolean circuit on a given input.

In terms of time complexity, it can be solved in linear time (topological sort).
\end{CountingDefinition}

\begin{itemize}
\item P-complete
\item limit of power of reductions
\item got a little tired and zoned out
\end{itemize}

\subsection*{The reduction (?)}


\begin{CountingDefinition}[Problem: CIRCUIT SAT]{def:validLabelPlacement}
The circuit satisfiability problem (CIRCUIT SAT) is the decision problem of determining whether a given Boolean circuit has an assignment of its inputs that makes the output true.
\ \\

Input: a Boolean circuit $C$ 
\ \\

Question: Is there a truth assignment which makes $C$ output the value true?
\end{CountingDefinition}

\subsection*{CIRCUIT SAT is NP-complete}

\begin{itemize}
\item circuit decides nondeterministically (?)
\item a variable is added n the nondeterministic Turing Machine
\item check if one of the variables is tue: use this choice (?)
\item problem: can we set thiese variables such that the Turing Machine accepts?
\item answer corresponds direct to \textit{is there a choice of nd decisions such that the turing machine accepts?}
\item extremely direct reduction
\end{itemize}




\color{red} TODO \\
font for caps lock \\
\color{black}
\color{violet} Questions:
\color{black}


\newpage

\printbibliography




\end{document}