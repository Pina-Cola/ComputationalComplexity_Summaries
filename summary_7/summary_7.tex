\documentclass[a4]{scrartcl}

% \usepackage[ngerman]{babel}
\usepackage[utf8]{inputenc}
\usepackage{mathtools}
\usepackage{amsmath}
\usepackage{amssymb}
\usepackage{geometry}
\usepackage{scrlayer-scrpage}
\usepackage{float}
\usepackage{xcolor}
\pagestyle{scrheadings}
\clearscrheadfoot

\usepackage[backend=biber, maxbibnames=99]{biblatex}
\addbibresource{references.bib}

\setlength{\parindent}{0cm}


\geometry{
  paper=a4paper, % Change to letterpaper for US letter
  top=2cm, % Top margin
  bottom=1.5cm, % Bottom margin
  left=2cm, % Left margin
  right=3cm, % Right margin
}

\ohead{\\
Pina Kolling\\
piko0011}

\usepackage[framemethod=TikZ]{mdframed}

% Style %
\mdfdefinestyle{enviStyle}{
   innertopmargin = 10pt,
  linewidth      = 1pt,
  frametitlerule = true,
  roundcorner    = 2pt%
}


\newenvironment{CountingDefinition}[2][]{%
   \ifstrempty{#1}%
   {\mdfsetup{%
      frametitle={{\strut ~}}}
   }%
   {\mdfsetup{%
      frametitle={{\strut ~#1}}}%
   }%
   \mdfsetup{
      nobreak                   = true,
     linecolor                 = gray,
    frametitlebackgroundcolor = gray!50,
    style                     = enviStyle
   }
   \begin{mdframed}[]\relax%
   \label{#2}}{\end{mdframed}}

\begin{document}

\section*{Summary: Lecture 7}

Summary for the chapters \textit{9.1} and \textit{9.2}. \cite{book, CC}

\subsection*{NP-Completeness}

\begin{CountingDefinition}[NP]{def:validLabelPlacement}
Class of lanugages decided by nonderteministic Turing machines in polynomial time.

Most problems are in NP.
\end{CountingDefinition}

\textbf{NP-completeness:}
\begin{itemize}
\item easiest problems among those we do not know how to solve efficiently
\item if P$\neq$NP can be proven: exact border of efficient solvability is found
\item best bet for proving P$=$NP: show that some NP-complete problem is $P$
\item Until then, the NP-complete problems are the least likely ones in NP to be efficiently solved
\item Where is the line between P and NP?
\end{itemize}



\begin{CountingDefinition}[Lanugage $L$]{def:validLabelPlacement}
$L= \{ x: (x,y) \in R \textit{ for some } y \}$ \ 
\\
$L$ gets an input $x$ and finds a $y$ with $((x,y) \in R$ and the relation $R \subseteq \Sigma^* \times \Sigma^*$.
\end{CountingDefinition}


\textbf{Polynomially decidable:}
\begin{itemize}
\item $R$ is polynomially decidable if there is da deterministic Turing machine deciding the language $L$ in polynomial time
\item then the relation $R$ (not the language $L$) is polynomially decidable
\end{itemize}

\textbf{Polynomially balanced:}
\begin{itemize}
\item $R$ is polynomially balanced if $(x,y) \in R$ implies $|y| \leq |x|^k$ for some $k \geq 1$ \\
$\rightarrow$ length of the second component is bounded by a polynomial in the length of the first
\item then the relation $R$ (not the language $L$) is polynomially balanced
\end{itemize}

\begin{CountingDefinition}[NP]{def:validLabelPlacement}
The language $L \subseteq \Sigma^*$ is in NP only if there is a polynomially decidable and polynomially balanced relation $R$ such that  $L= \{ x: (x,y) \in R \textit{ for some } y \}$.
\end{CountingDefinition}

For example: Is there a satisfying assignment ($y$) for a formular ($x$)?
\color{violet} \\
Why is $R \subseteq \Sigma^* \times \Sigma^*$? Is the input formula and the truth assignment $\in \Sigma^*$?
\color{black}

\ \\ \ \\

\color{red} TODO \\
proof \\
\color{black}
\color{violet} Questions: \\
\color{black}

\subsection*{Succinct certificate (for NP-complete problems)}

\begin{itemize}
\item \textit{yes} instance of $x$ has a polynomial witness $y$ (certificate)
\item \textit{no} instances don't have such a certificate
\item Examples:
\begin{itemize}
\item \textsc{Sat}: certificate is the truth assignment
\item \textsc{HamiltonPath}: certificate is the hamilton path of a graph
\end{itemize}
\end{itemize}


\subsection*{Typical problems in NP}

\begin{itemize}
\item sometimes the optimum needs to be found
\item sometimes any object that fits the specification is enough
\item constraints can be added to optimization problems
\end{itemize}



\subsection*{\textsc{3Sat} is NP-complete}


\begin{CountingDefinition}[\textsc{SAT}]{def:validLabelPlacement}
The \textsc{SAT} (satisfiability) problem is the problem of determining if there exists an interpretation that satisfies a given Boolean formula. \cite{GTI}
\end{CountingDefinition}


\begin{CountingDefinition}[\textsc{3SAT}]{def:validLabelPlacement}
Like the \textsc{Sat} problem, \textsc{3Sat} is determining the satisfiability of a formula in CNF where each clause is limited to at most three literals.
\end{CountingDefinition}

\begin{itemize}
\item $k$\textsc{Sat} with $k \geq 1$ is a special case of \textsc{Sat}
\end{itemize}

\textbf{Reduction from} \textsc{Sat} \textbf{to} \textsc{3Sat}: \cite{sat}
\begin{itemize}
\item the reduction replaces each clause with a set of clauses, each having exactly three literals
\item rewrite the clauses of of the input
\item example:
\begin{align*}
\ & (x_1) \land (x_1 \lor \bar{x_2}) \land (x_2 \lor x_3 \lor x_5) \land (x_1 \lor x_4 \lor \bar{x_6} \lor \bar{x_7}) \land (x_1 \land x_2 \land \bar{x_3} \lor x_5 \lor x_7) \\
\equiv \  & (x_1 \lor x_1 \lor x_1) \land (x_2 \lor x_3 \lor x_5)
\end{align*}
\end{itemize}

\subsection*{\textsc{3Sat} with more retrictions}


\color{red} TODO \\
\color{black}
\color{violet} Questions:
\color{black}

\subsection*{\textsc{2Sat} in P (graph construction)}

\begin{CountingDefinition}[Title]{def:validLabelPlacement}
Content
\end{CountingDefinition}

\color{red} TODO \\
\color{black}
\color{violet} Questions:
\color{black}

\subsection*{\textsc{2Sat} in NL}

\begin{CountingDefinition}[Title]{def:validLabelPlacement}
Content
\end{CountingDefinition}

\color{red} TODO \\
\color{black}
\color{violet} Questions:
\color{black}

\subsection*{\textsc{MaxSat} is NP-complete}

\begin{CountingDefinition}[Title]{def:validLabelPlacement}
Content
\end{CountingDefinition}

\color{red} TODO \\
\color{black}
\color{violet} Questions:
\color{black}


\subsection*{\textsc{NaeSat} is NP-complete}

\begin{CountingDefinition}[Title]{def:validLabelPlacement}
Content
\end{CountingDefinition}

\color{red} TODO \\
\color{black}
\color{violet} Questions:
\color{black}



\newpage

\printbibliography




\end{document}