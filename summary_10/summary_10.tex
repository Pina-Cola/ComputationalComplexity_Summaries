\documentclass[a4]{scrartcl}

% \usepackage[ngerman]{babel}
\usepackage[utf8]{inputenc}
\usepackage{mathtools}
\usepackage{amsmath}
\usepackage{amssymb}
\usepackage{geometry}
\usepackage{scrlayer-scrpage}
\usepackage{float}
\usepackage{xcolor}
\pagestyle{scrheadings}
\clearscrheadfoot

\usepackage[backend=biber, maxbibnames=99]{biblatex}
\addbibresource{references.bib}

\setlength{\parindent}{0cm}


\geometry{
  paper=a4paper, % Change to letterpaper for US letter
  top=2cm, % Top margin
  bottom=1.5cm, % Bottom margin
  left=2cm, % Left margin
  right=3cm, % Right margin
}

\ohead{\\
Pina Kolling\\
piko0011}

\usepackage[framemethod=TikZ]{mdframed}

% Style %
\mdfdefinestyle{enviStyle}{
   innertopmargin = 10pt,
  linewidth      = 1pt,
  frametitlerule = true,
  roundcorner    = 2pt%
}


\newenvironment{CountingDefinition}[2][]{%
   \ifstrempty{#1}%
   {\mdfsetup{%
      frametitle={{\strut ~}}}
   }%
   {\mdfsetup{%
      frametitle={{\strut ~#1}}}%
   }%
   \mdfsetup{
      nobreak                   = true,
     linecolor                 = gray,
    frametitlebackgroundcolor = gray!50,
    style                     = enviStyle
   }
   \begin{mdframed}[]\relax%
   \label{#2}}{\end{mdframed}}

\begin{document}
















%--------------------------------------------------------------------------------------------------------------




\section*{Summary: Lecture 10}

Summary for the chapters \textit{11.1} up to page 245 and \textit{11.2} (page 258 optional). \cite{book, CC}

\begin{CountingDefinition}[Randomized algorithms]{def:validLabelPlacement}
Algorithms based on randomization.


(The algorithm employs a degree of randomness as part of its logic or procedure.)
\end{CountingDefinition}

















%--------------------------------------------------------------------------------------------------------------



\subsection*{Bipartite matching}

\begin{CountingDefinition}[Bipartite Graph]{def:validLabelPlacement}
A graph $G = (U,V,E)$ is called bipartite if the vertices can be divided into two disjoint and independent sets $U$ and $V$. (There are no edges between two elements of $U$ or two elements of $V$).

\begin{figure}[H]
\begin{center}
\includegraphics[scale=0.12]{bg1.png}
\includegraphics[scale=0.12]{bg2.png}
\end{center}
\caption{Examples of bipartife graphs with $U$ and $V$ marked in red and blue \cite{bg}}
\end{figure}

\end{CountingDefinition}

\begin{CountingDefinition}[Problem: BipartiteMatching]{def:validLabelPlacement}
Given: Bipartite graph $G=(U,V,E)$.

\ \\
Is there a perfect matching $M \subseteq E$ such that for any two edges $(u,v)$ and $(u', v')$ in $M$ $u \neq u'$ and $v \neq v'$.

\ \\
I other words: A matching in a Bipartite Graph is a set of the edges chosen in such a way that no two edges share an endpoint. The matching $M$ is called perfect if for every node in $V$ there is some edge in $M$. 
\cite{bm, book, bgm}
\end{CountingDefinition}

\begin{itemize}
\item construct bipartite graph with $n$ nodes as $n \times n$ matrix $A$
\item the element $A_{i,j}$ is a variable $x_{i,j}$ if $(i,j) \in E$
\item the element $A_{i,j}$ is 0 if $(i,j) \notin E$
\end{itemize}

\textbf{Example: }

\begin{minipage}{0.50\textwidth}

\begin{center}
\includegraphics[scale=0.28]{v1.png}
\end{center}

\end{minipage}
\begin{minipage}{0.50\textwidth}


$ A = \begin{pmatrix}
0 & x_{1,2} & 0 & x_{1,4} \\
x_{2,1} & 0 & x_{2,3} & 0 \\
0 & x_{3,2} & 0 & x_{3,4} \\
x_{4,1} & 0 & x_{4,3} & 0
\end{pmatrix}$

\end{minipage}

\color{violet} Questions: \ \\
Is EVERY node cotained in the subset $M$ when it is a perfect matching? Does a perfect matching then only exist with an even number of vertices and $|U| = |V|$? \ \\
\color{black}
















%--------------------------------------------------------------------------------------------------------------




\subsection*{Determinant calculation}

\textbf{Leibniz-formula:}
\ \\ \ \\
\begin{minipage}{0.1\textwidth}

\ 


\end{minipage}\begin{minipage}{0.3\textwidth}
\ \\
$ A = \begin{pmatrix}
a & b & c  \\
d & e & f \\
g & h & i
\end{pmatrix}$

\end{minipage}
\begin{minipage}{0.5\textwidth}

$|A| = a \cdot e \cdot i +  b \cdot f \cdot g + c \cdot d \cdot h - g \cdot e \cdot c - h \cdot f \cdot a - i \cdot d \cdot b$


\end{minipage}

\ \\ \ \\
\begin{minipage}{0.1\textwidth}

\ 


\end{minipage}\begin{minipage}{0.3\textwidth}
\ \\
$ B = \begin{pmatrix}
1 & 2 & 3  \\
4 & 5 & 6 \\
7 & 8 & 9
\end{pmatrix}$

\end{minipage}
\begin{minipage}{0.5\textwidth}

$|B| = 1 \cdot 5 \cdot 9 +  2 \cdot 6 \cdot 7 + 3 \cdot 4 \cdot 8 - 7 \cdot 5 \cdot 3 - 8 \cdot 6 \cdot 1 - 9 \cdot 4 \cdot 2 = 0$


\end{minipage}

\begin{align*}
\\
|A| = \sum_{\pi} \sigma(\pi) \prod^{n}_{i=1} A_{i, \pi(i)}
\end{align*}
\begin{itemize}
\item $\sigma(\pi)$ decides if $+$ or $-$
\item leads to $n!$ summands
\item Example: $n= 3$ \\
$3! = 6$ summands \\
$6$ permutations for $\pi$ 
\begin{itemize}
\item[$+1$: ] $( \ 1 \ 2 \ 3 \ ) \ \ ( \ 2 \ 3 \ 1 \ ) \ \  ( \ 3 \ 1 \ 2 \ ) $
\item[$-1$: ] $( \ 3 \ 2 \ 1 \ ) \ \ ( \ 2 \ 1 \ 3 \ ) \ \  ( \ 1 \ 3 \ 2 \ ) $
\item[$\rightarrow$] $|A| = A_{1,1} \cdot A_{2,2} \cdot A_{3,3} + A_{2,1} \cdot A_{3,2} \cdot A_{1,3} + ...$
\end{itemize}
\end{itemize}



\textbf{Gaussian elimination:}
\begin{itemize}
\item Gauß algorithm for solving LSE (linear systems of equations)
\item allowed operations:
\begin{itemize}
\item addition of rwos
\item subtraction of rows
\item multiply row with integer $x$
\item divide row by integer $x$
\item switch to rows
\end{itemize}
\item wanted: upper triangular form (all entries below the diagonal 0)
\item determinant is the product of the diagonal entries
\item Example:
\begin{figure}[H]
\begin{center}
\includegraphics[scale=0.4]{gauss.jpg}
\end{center}
\caption{Examples gaussian elimination \cite{book}}
\end{figure}
$|A| = 1 \cdot 4 \cdot 7 \cdot 7 = 196$
\end{itemize}



















%-----------------------------------------------------------------------------------------------------------------------------








\subsection*{Symbolic Determinants}



\ \\

\color{red} TODO \\
\color{black}
\color{violet} Questions: \\
\color{black}


















%-----------------------------------------------------------------------------------------------------------------------------






\subsection*{Monte Carlo algorithm}



\ \\

\color{red} TODO \\
\color{black}
\color{violet} Questions: \\
\color{black}



















%-----------------------------------------------------------------------------------------------------------------------------






\subsection*{Randomized complexity classes}



\ \\

\color{red} TODO \\
\color{black}
\color{violet} Questions: \\
\color{black}




\begin{itemize}
\item some slides about stuff that is not in the book
\item Satz von Rice
\end{itemize}


\color{red} TODO \\
\color{black}
\color{violet} Questions: \\
\color{black}



\newpage

\printbibliography




\end{document}